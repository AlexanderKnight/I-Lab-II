\documentclass[aps,secnumarabic,balancelastpage,amsmath,amssymb,nofootinbib]{revtex4-1}
\usepackage{graphicx}
\usepackage{amssymb}
\usepackage{hyperref}
\textwidth = 6.25 in
\textheight = 8.5 in
\oddsidemargin = 0.5 in
\evensidemargin = 0.5 in
\topmargin = 0.0 in
\headheight = 0.0 in
\headsep = 0.2 in
\parskip = 0.2in
\parindent = 0.0in

\begin{document}\thispagestyle{empty}
\begin{center}
{\large \bf Data Plotting and Fitting Exercise}
\rule{\columnwidth}{0.5pt}\\[-15mm]
\end{center}
\vspace*{-2mm}
For most research projects, you will have to plot data and 
find and plot theoretical models to compare with that data. 
Powerful computer packages such as Mathematica, Matlab,
Origin, Igor Pro, and Python/NumPy/Matplotlib are now available and 
commonly used in research labs for these purposes. 
The purpose of this exercise is to get you to learn the rudiments 
of one of these packages, if you don't already know them.

The assignment is divided into stages of increasing difficulty and sophistication. 
I want you to use a full---featured scientific data-handling package; either\\[-8mm]
\begin{enumerate}
\item Python/NumPy/Matplotlib (open source)\\ Requires full python distribution---I recommend the Anaconda Python Distribution)
\item Igor Pro (commercial; Wavemetrics, Inc.)\\ Serial Number: 37955\\Activation Key: ZMAT-TFLA-FWYY-NZYX-AX
\end{enumerate} 
\vspace*{-5mm}
All of these questions use the "dummy data" that is listed at the end of this document (and is 
available as a text file at \href{http://people.usm.maine.edu/pauln/ilab-phy240/}{http://people.usm.maine.edu/pauln/ilab-phy240/} .)

I do not need a lot of writing for this assignment, but I would like you to write your solutions in an iPython Notebook using a combination of markdown cells (using  \LaTeX\ code where needed) and python commands that produce the required plots.  
Make sure to include your name, date, and the actual plots with labelled axes, units, and anything else you
need. Please turn in a printed copy of your solutions, and post an iPython notebook to your github site (make sure your notebook printout
includes the github address). 

You may find an example of fitting data using python at \href{http://scipyscriptrepo.com/wp/?p=76}{http://scipyscriptrepo.com/wp/?p=76}.

\textbf{5 Questions; Due:} BEFORE class Monday, 1 February 2016\\
\rule{\columnwidth}{0.5pt}
\vspace*{-9mm}
\begin{enumerate}
\item \textsf{Simple plot and straight-line fit.} \\
Plot the $z$ data set versus the $x$ data set on linear axes, using symbols.\\
(not a curve through the data points, not a set of straight line segments, and not tiny invisibly small points). 
Be sure the axes are labeled and you include units.
Then, determine the best straight-line fit to the data and produce a plot showing both the data and the fit.

\item \textsf{Semi-log and log-log plots and what they tell us.}\\ Plot the $u$ data set versus the $x$ data set. 
Make one, the other, or both axes logarithmic- do any of these turn the plot into a straight line? 
If so, what does that tell us about the data (what kind of function? what power?). 
Repeat with the $v$ data set plotted versus $x$.

\item \textsf{Plot data with error bars.}\\
Plot the y data set versus the $x$ data set, using the $e$ data set as uncertainties on $y$. 
That is, each data point should show a symbol at $(x_i, y_i)$ along with an error bar extending by $\pm e_i$ in the y direction.

\item \textsf{Linear least-squares fit.}\\ Find the best fit of a quadratic function to the $y$ versus $x$ data. That is, find the values of $a$, $b$, $c$ 
such that $y = a x^2 + b x + c$ best describes the data. This will normally involve minimizing the sum of the squares of the differences 
between the fit and the data, hence the name "least squares". Plot the fit and the data (with error bars) on the same graph, 
and print out the values of $a$, $b$, $c$. Usually it works best to plot the data using points or other symbols, and the fit as a smooth curve.
(Do not connect the points by a jagged line that runs point-to-point.)
\pagebreak
\item \textsf{Nonlinear least-squares fits.}\\
 Find and plot the best fit of the $y$ versus $x$ data to a gaussian function: 
$$ y = a e^{-\frac{(x-c)^2}{2\sigma^2}}, $$
with $a$, $c$, and $\sigma$ to be determined. Do the same thing for a cosine function: 
$$ y = a \cos(\omega(x-c)).$$
Note: This question is "harder" (requires a more sophisticated fitting program) than question four, because the gaussian and cosine 
functions are not simply linear combinations of given functions.


\end{enumerate}



	\begin{table}[h]
	\caption{\label{tab:stone} Fake Data set for this assignment.}
	\begin{ruledtabular}
	\begin{tabular}{rrrrrr}
	 x 	&	z	&	u	&	v	&	y	&	e \\
	\hline
	(s) 	&	(m)	&	(m)	&	(m)	&	(m)	&	(m)\\
	\hline
0.015	&	0.921	&	0.212	&	1.381	&	0.015	&	0.070 \\
0.172	&	1.472	&	0.457	&	1.328	&	0.210	&	0.109 \\
0.329	&	1.527	&	0.598	&	1.731	&	0.440	&	0.187 \\
0.486	&	1.921	&	0.703	&	1.987	&	0.547	&	0.140\\
0.643	&	2.792	&	0.815	&	1.964	&	0.764	&	0.235\\
0.800	&	3.181	&	0.909	&	2.644	&	0.753	&	0.105\\
0.958	&	3.869	&	0.998	&	2.864	&	0.960	&	0.212\\
1.115	&	3.955	&	1.060	&	3.262	&	0.959	&	0.131\\
1.272	&	4.081	&	1.139	&	4.042	&	0.974	&	0.088\\
1.429	&	5.127	&	1.210	&	4.448	&	1.019	&	0.099\\
1.586	&	5.133	&	1.262	&	5.119	&	1.198	&	0.268\\
1.743	&	5.906	&	1.323	&	6.138	&	1.009	&	0.094\\
1.900	&	5.709	&	1.391	&	6.914	&	0.948	&	0.072\\
2.057	&	6.447	&	1.442	&	8.315	&	0.990	&	0.176\\
2.214	&	7.230	&	1.504	&	9.524	&	0.920	&	0.190\\
2.317	&	7.951	&	1.542	&	10.809	&	0.730	&	0.103\\
2.528	&	8.070	&	1.603	&	12.953	&	0.666	&	0.160\\
2.685	&	8.800	&	1.642	&	14.915	&	0.452	&	0.081\\
2.843	&	8.986	&	1.698	&	17.173	&	0.451	&	0.227\\
3.000	&	9.743	&	1.743	&	20.364	&	0.245	&	0.174\\
	
	
	
	\end{tabular}
	\end{ruledtabular}
	\end{table}











\end{document}
